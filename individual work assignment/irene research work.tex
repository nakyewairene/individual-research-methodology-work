

\documentclass[10pt,]{article}
\usepackage{zed-csp,graphicx,color}%from
\pagenumbering{roman}
\begin{document}
\begin{titlepage}
\centerline{The effects of stress on employees and programs offered by employers to manage employee stress in kampala, uganda.\\}
\paragraph*{•}
\centerline{  Prepared by:  Nakyewa irene 16/U/851 216000677.\\}
\paragraph*{•}
\paragraph*{•}
  \begin{flushright}
  The Report,\\
  DATE: $February,6^{th},2018$.
 \tableofcontents

  \end{flushright}
\date{\today}
\end{titlepage}

\newpage



\pagenumbering{arabic}
\section{Introduction}
Today, many organizations in Uganda and employees are experiencing the effects of stress on work performance. The effects can either be positive or negative. According to research made negative stress is becoming the major illness in the work environment, and it can deliberate employees and be costly to employers. managers need to identify those suffering from negative stress and implement programs as a defense against stress.

\section{Statement of the problem}
The purpose of this study was to determine the negative effects of stress on employees and the methods employers use to manage employee stress

\section{Main objectives of the study}
\begin{itemize}
\item The main objective to examine the nature of job stress in Uganda and to identify the factors responsible for job stress impact on employee’s productivity and performance in Uganda.
\item To ascertain the strategies for dealing with job stress among workers in Uganda.
\end{itemize}


\section{Significance of the study}
There are several groups of employees that benefit from this study, the first group consisting of employees in today’s business organizations, may learn to identify ways that stress negatively affects their work performance. identifying the negative effects may enable them to take necessary action to cope with stress. Then educators can use these findings as valuable guide to incorporate unto their curriculum.
\section{Scope of the study}
This study was limited to perceptions of fulltime business employees as to the negative effects that stress has on work performance and the steps that employees are taking to manage stress, this study was restricted to businesses operating in rural areas outside Kampala.
\section{Limitations of the study}
This study was limited through the use of questioniare as a data collection instrument. Because questionnaires must generally be brief ,areas that may have been affected by stress may not have been included in the questioniare.Also all programs that may be available to employees for managing stress may not have been included in the study .The study may also be limited by the use of a nonprofit ability, convenience sampling method. 
The sample of business employees for study was chosen for conveniences and may not be representative of the total population of business employees .Finally the use of sampling statistical techniques may introduce an element of subjectivity into the interpretation and analysis of the data
\section{Methods of study used }
 \subsection{Source of data }
The research was collected by description through questionnaires and sampling a group of employees and employers in some companies located around Kampala
 \subsection{Sample selection }
The respondents involved in this survey were employees working in companies in Kampala areas, a anon profitability, convenience sampling technique was used to collect primary data ensure confidentiality respondents were given questioners and controls were used to eliminate duplication of response. And from the study it can be concluded that employees have realized the importance of managing stress in the work place because of the wide variety of programs now offered to manage stress
\section{Recommendations }
Based on the findings and conclusions in this study, the following recommendations are made:
\begin{enumerate}
\item Employers should offer various stress reduction programs to help employees manage stress because stress is prevalent in the work place.
\item Employers should conduct a survey of the programs they already offer to discover which programs are the most effective for managing their employees stress
\item Employers should incorporate into their business curriculum discussions of stress in the work places to manage stress
\end{enumerate}
\section{References}
\begin{itemize}
\item Graham Birley and Neil Moreland, 1998 A Practical Guide to Academic Research ,www.job-sift.com/Employment Stress 
\end{itemize}

\end{document}






